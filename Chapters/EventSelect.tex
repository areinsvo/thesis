\chapter{Sample Selection}
\label{chap:EventSelect}

\section{Object Selection}
\label{sec:ObjSelect}
In addition to photons and electrons, a third, orthogonal object definition referred to as ``fake" photons is also used. 
Fakes are primarily electromagnetically-rich jets that have been misidentified as photons. As will be described in Chapter~\ref{chap:DataAnalysis},
a control sample made up of events with two fakes is used to estimate the QCD background. 

\subsection{Photon identification}
\label{sec:phoID}

\subsection{Electron identification}
\label{sec:eleID}


\subsection{Fake identification}
\label{sec:fakeID}
Electromagnetically-rich jets that have been misidentified as photons make up the majority of the objects 
that satisfy our ``fake" definition. These objects are taken from a photon ID sideband. Fakes are required to 
pass all of the photon identification criteria described in Section~\ref{sec:phoID}, except they are required to
fail either the \sigmaietaieta or the charged hadron isolation requirement. This is described in detail in Table~\ref{tab:ID}
(XX better if column widths the same XX)

\begin{table}[ht]
    \caption{ELECTROMAGNETIC OBJECT DEFINITIONS}
    \centering
    \begin{tabular}{ | c | c c c |}
        \hline
        	\hline
        \textbf{ID Requirement} & \textbf{Photons} & \textbf{Electrons} & \textbf{Fakes} \\ [0.5ex]
        \hline
        	Pixel seed veto    & \multicolumn{1}{c|}{Applied} & \multicolumn{1}{c|}{Reversed} & \multicolumn{1}{c|}{Applied}\\
	\hline
	\sigmaietaieta   & \multicolumn{2}{c|}{$ < 0.01022 $} & $0.01022 < \sigmaietaieta < 0.015 $\\
	Charged hadron isolation & \multicolumn{2}{c|}{$ < 0.441$} & $ 0.441 < iso < 15$\\
	\hline
	Photon isolation & \multicolumn{3}{c|}{$ < 2.571 +0.0047~\pt$} \\
	Neutral hadron isolation   &  \multicolumn{3}{c|}{$ < 2.2725 + 0.0148~\pt+0.000017~\pt^2$} \\
        R9                      & \multicolumn{3}{c|}{$ > 0.5$} \\
        H/E                     & \multicolumn{3}{c|}{$ < 0.0396$} \\
           \hline
           \hline
    \end{tabular}
    \label{tab:ID}
    \justify{Definitions of photon, electron, and fakes used to define the signal and control samples for this analysis. ``Fakes" refer to 
    jets that have been misidentified as photons. The definitions of each of the variables used in the object ID's can be found in 
    Section~\ref{sec:Var}.}
\end{table}

STOLEN %https://twiki.cern.ch/twiki/bin/viewauth/CMS/CutBasedPhotonIdentificationRun2#Selection_implementation_details
The uncorrected isolation variables are PF objects 
Pt sums in a cone 0.3 around the photon and also subjected to 
the photon footprint removal using the association map of the PF objects. 
The charge hadron Isolation is calculated using the PF objects that their tracks are associated with the primary vertex only, thus removing pile-up charge hadrons.
STOLEN 

\begin{table}[ht]
    \caption{EFFECTIVE AREAS FOR ISOLATION CORRECTIONS}
    \centering
    \begin{tabular}{ | c | c c c |}
        \hline
        	\hline
        \textbf{$|\eta|$ Range} & \textbf{Photon Iso} & \textbf{Neutral Hadron Iso} & \textbf{Charged Hadron Iso} \\ [0.5ex]
        \hline
        	$|\eta| < 1.0 $                 & 0.120   &  0.0597 & 0.0360\\
	$ 1.0 < |\eta| < 1.479 $   & 0.1107 & 0.0807 & 0.0377 \\
		 \hline
           \hline
    \end{tabular}
    \label{tab:ID}
    \justify{Effective areas used in the definition of photon, charged hadron, and neutral hadron isolation values. }
\end{table}



\subsection{Object Cleaning}
\label{sec:ObjCleaning}

In order to avoid double counting particles, a set of object cleaning rules are applied. 
First, because muons are reconstructed with a higher purity than any other particle, 
any electromagnetic object (photon, electron, or fake) that is within \dR $< 0.3$ of a muon candidate is removed.
Second, 

all photon, electron, and fake candidates are 

\subsection{Lepton veto}
\label{sec:lepVeto}



