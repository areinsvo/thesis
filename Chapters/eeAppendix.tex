\chapter{ALTERNATE CONTROL SAMPLE FOR QCD BACKGROUND ESTIMATION}
\label{app:ee}
In the analysis performed using the 2015 CMS data set~\cite{CMS:2015_anal}, 
a sample of $Z\rightarrow ee$ events (referred to as the $ee$ sample) was used as the 
primary QCD background estimate, and the $ff$ control 
sample was used to set a systematic uncertainty on the shape of the \ETmiss tail.
The idea was that the $ee$ and $ff$ control samples represent
100$\%$ and 0$\%$ photon purity, respectively. By using both, we could
derive a bound on any possible sensitivity of the \ETmiss spectrum
to the object purity. 

\section{Problems with using $ee$}
\label{sec:eeProblems}
There are several reasons we decided not to use the $ee$ control sample
 in the 2016 analysis. One reason was uncertainty on 
 the efficiency of the control 
 trigger. As mentioned in Chapter~\ref{chap:Trigger},
the primary analysis trigger 
cannot be used for $Z\rightarrow ee$ events due to the $m_{\gamma\gamma}~>~90$ GeV
invariant mass requirement. Instead, the $ee$ sample uses the control trigger 
listed in Table~\ref{tab:triggers}. As discussed in Section~\ref{sec:eeEff}, the
control trigger is only 79.8\% efficient because it 
requires both electromagnetic objects to be matched to a 
pixel seed. There is a relatively 
large uncertainty on this efficiency because of its
slow turn-on. Our offline photon \pT threshold of 40 GeV is 
barely in the plateau, and the trigger continues to get more efficient 
as \pT increases. 

Another reason why we prefer $ff$ over $ee$ is that the kinematics
of $ff$ are closer to the kinematics of our SUSY signal events.
In $ee$ events, the two electromagnetic objects are obviously correlated, 
since they come from the decay of the $Z$ boson. In $ff$ and SUSY diphoton
events, on the other hand, the photons and fakes are uncorrelated. The result
is that the $ee$ and $\gamma\gamma$ \ETmiss distributions start to diverge
once the statistical uncertainties go down. 

The most significant reason for not using $ee$ was actually 
not the trigger or kinematics, but large contributions from processes with real \ETmiss 
 such as $t\bar{t}$ and $ZZ\rightarrow ee\nu\nu$. 
The idea of the QCD background estimation method is that we 
select control samples that do not have inherent \ETmiss. 
For the case of $ee$, however,
there are processes with inherent \ETmiss that contaminate
the sample. 
In \ttbar events where both tops decay leptonically,
there can be final states with two electrons and
multiple neutrinos. Similarly, $ZZ \rightarrow ee\nu\nu$
events will fall into our $ee$ category.
These contributions are modeled with MC and subtracted 
from the $ee$ distribution in data. 
The MC samples used are listed in
Table~\ref{tab:eeDatasets}.

\begin{table}[ht]
  \caption{BACKGROUND MC SAMPLES FOR EE SAMPLE}
  \centering
  \begin{tabular}{|>{\centering\arraybackslash}m{0.9\linewidth}|}
    \hline
    \hline
    /TT$\_$TuneCUETP8M2T4$\_$13TeV-powheg-pythia8/
    RunIISummer16MiniAODv2-PUMoriond17$\_$ 
    80X$\_$mcRun2$\_$asymptotic$\_$2016$\_$TrancheIV$\_$v6-v1/MINIAODSIM\\
    \hline
    /ZZTo2L2Nu$\_$13TeV$\_$powheg$\_$pythia8/
    RunIISummer16MiniAODv2-PUMoriond17$\_$ 
    80X$\_$mcRun2$\_$asymptotic$\_$2016$\_$TrancheIV$\_$v6-v1/MINIAODSIM\\
    \hline
    \hline
    \end{tabular}
    \label{tab:eeDatasets}
     \justify{MC samples used to remove contributions
     to the $ee$
    control sample
    from processes with real \ETmiss.}
\end{table}

Table~\ref{tab:subtract} shows the percent contribution
from these processes in the signal region. In some bins, 
95\% of the observed $ee$ events can be attributed to 
these real-\ETmiss processes.
As you can see in Figure~\ref{fig:eeUnweighted},
very few $ee$ events remain in the tail after 
performing this subtraction.


\begin{table}[ht]
     \caption{CONTRIBUTIONS TO EE SAMPLE}
     \centering % used for centering table                                                                                                                    
     \begin{tabular}{| c | c | c |} % centered columns (4 columns)                                                                                            
    \hline
     \ETmiss bin (GeV) & \ttbar Contribution & $ZZ$ Contribution\\ [0.5ex]
     \hline
     $100-115$ & 56.1\% & 4.4\% \\
$115-130$ & 82.8\% & 8.54\% \\
$130-150$ & 75.8\% & 9.9\% \\
$150-185$ & 80.8\% & 14.6\% \\
$185-250$ & 50.9\% & 16.8\% \\
$> 250$ & 33.4\% & 27.7\% \\
     \hline
     \end{tabular}
     \label{tab:subtract}
\end{table}

\begin{figure*}[h]
\begin{center}
\includegraphics[width=0.75\textwidth]{Figures/Appendix/compare_ee_gg_unweighted.pdf}
\end{center}
\caption{Unweighted \ETmiss distributions of the $ee$ control sample and the $\gamma\gamma$ candidate sample.
The $ee$ distribution has been normalized to the \ETmiss$<50$ GeV region of the candidate sample. Contributions 
from \ttbar and $ZZ$ events have been subtracted.}
\label{fig:eeUnweighted}
\end{figure*}

%\subsection{Systematic uncertainties on \ttbar subtraction}
%\label{sec:ttbarUncert}

For the subtraction of $ZZ \rightarrow ee\nu\nu$ events, 
we consider uncertainties from
MC statistics and jet energy scale corrections.
Both uncertainties are less than a 3\% effect
on the final $ee$ estimate in the signal region.

The systematic uncertainties associated with 
the subtraction of the \ttbar contribution to the $ee$ control
sample are much more significant. 
Because we are subtracting a large majority of the
original $ee$ events at high \ETmiss, the relative
uncertainties on the expected number of $ee$ events get inflated.
In the signal region, the limited MC \ttbar statistics
leads to an uncertainty of up to 140\%. Uncertainties
from the parton distribution function and overall
scale of the cross section lead to uncertainties
on the final $ee$ estimate up to 73\%.
Jet energy scale corrections
applied to the \ttbar sample lead to uncertainties up to
60\%, and the effect of reweighting by the
top \pt to correct for differences between
MC and data is up to 70\%.

The large corrections and associated uncertainties
to make the $ee$ sample representative of the
events we were trying to model
was the primary reason why we ultimately
decided not to use the $ee$ control sample for the
QCD background estimate.

%%%%%%%%%%%%%%%%%%%%%%%%%%%%%%%%%%%%%%%
%%%%%%%%%%%%%%%%%%%%%%%%%%%%%%%%%%%%%%%

\section{Results using ee sample}
\label{sec:eeResults}
As was done for the $ff$ sample,
any difference in hadronic activity between the control and candidate
samples needs to be corrected.
Figure~\ref{fig:eediEMPTRatio} shows the $ee$ and 
$\gamma\gamma$ di-EM \pt distributions. 
The \ETmiss distributions after reweighting are shown in Figure~\ref{fig:eeWeighted}.
It is clear from the plot that even after \diempt reweighting, the $ee$ sample 
systematically underestimates the QCD background.

\begin{figure*}[h]
\begin{center}
\includegraphics[width=0.75\textwidth]{Figures/Appendix/eeDiempt.pdf}
\end{center}
\caption{Comparison of the di-EM \pt distributions of the $\gamma\gamma$ and $ee$ samples.}
\label{fig:eediEMPTRatio}
\end{figure*}

\begin{figure*}[h]
\begin{center}
\includegraphics[width=0.75\textwidth]{Figures/Appendix/compare_ee_gg_unweighted.pdf}
\end{center}
\caption{\ETmiss distributions of the candidate sample and the reweighted $ee$ sample.
The $ee$ distribution has been normalized to the \ETmiss$<50$ GeV region of the candidate sample.
Contributions to $ee$
from \ttbar and $ZZ$ events have been subtracted.}
\label{fig:eeWeighted}
\end{figure*}

%This includes differences in hadronic
%activity as modeled by the di-EM \pt variable, but also includes
%differences in the jet multiplicity distributions.
%For example, the energy resolution in an event where
%there are three jets having a total \pt of $100$ GeV will be worse than that
%of an event where there is only one jet with \pt of $100$ GeV.
%%\begin{figure}
%%\begin{center}
%% \includegraphics[width=6.0in]{figs/JetMultiplicity.pdf}
%% \end{center}
%% \caption{Jet multiplicity distributions of the candidate, $ee$ and $ff$ samples.}
%% \label{fig:JetDistribution1D}
%%\end{figure}
%
%To account for any possible dependence between the di-EM \pt and the
%jet multiplicity, the $ee$ sample was reweighted by the ratio of
%$\gamma\gamma$ over $ee$ in bins of jet multiplicity versus
%di-EM \pt. The reweighting factors are shown in Figure \ref{fig:JetDistribution2D}.
%Figure~\ref{fig:eeReweighting} shows the effect of reweighting
%the $ee$ control sample by di-EM \pt only or by reweighting the
%$ee$ \ETmiss distribution by
%the jet multiplicity in bins of di-EM \pt.
%
%%\begin{figure}
%%\begin{center}
%% \includegraphics[width=6.0in]{figs/NJetVsDiempt.pdf}
%% \end{center}
%% \caption{$\gamma\gamma$ over $ee$ jet multiplicity distribution ratio in bins of di-EM \pt.}
%% \label{fig:JetDistribution2D}
%%\end{figure}
%%
%%\begin{figure}
%%\begin{center}
%% \includegraphics[width=6.0in]{figs/NJetMetCompare.pdf}
%% \end{center}
%% \caption{\ETmiss distributions for the $ee$ sample reweighted by di-EM \pt only (red) and the $ee$ sample reweighted by the 2D distribution of di-EM
%%\pt vs jet multiplicity (black). The ratio plot shows the ratio of red points to black points.}
%% \label{fig:eeReweighting}
%%\end{figure}
%
%A comparison of the unweighted \ETmiss distributions for the $ee$ and
%$ff$ control samples and the $\gamma\gamma$ candidate sample
%is shown in Figure \ref{fig:eeUnweighted}. The comparison after
%the reweighting procedure is shown in Figure~\ref{fig:eeWeighted}.
