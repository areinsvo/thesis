\chapter{THE STANDARD MODEL AND SUPERSYMMETRY}
\label{chap:theory}

\section{The standard model of particle physics}
\label{sec:StandardModel}
The standard model (SM) of particle physics is a non-Abelian gauge theory that seeks to describe and quantify everything that is known about elementary particles and their allowed interactions. The current form of the standard model has passed every experimental test to an incredible level of precision. The first piece of the standard model was developed in 1961 (XX references! XX) with the unification of the electromagnetic and weak interactions. The Higgs mechanism was later incorporated into the standard model in 1967 (XX cite XX), and the standard model took on the form we know today with the inclusion of the strong force and quantum chromodynamics (QCD) in the 1970's (XX cite XX).

The standard model is a Lorentz-invariant quantum field theory. The symmetry group of the standard model is 
\begin{equation}
SU(3)_c \oplus SU(2)_L \oplus SU(1)_\Upsilon
\end{equation}
where $SU(3)_c$ represents the QCD


\section{Supersymmetry}
\label{sec:SUSY}

Motivations:
Higgs mass, contributions 
Hierarchy problem 

\section{Gauge-mediated supersymmetry breaking}
\label{sec:gmsb}

