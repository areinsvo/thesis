\chapter{Data analysis and background estimation methods}
\label{chap:DataAnalysis}

\section{Overview}

There are several Standard Model processes that can mimic our signal events. The largest background contribution comes from quantum chromodynamics (QCD) processes. These are primarily multi-jet events, where electromagnetically-rich jets are misidentified as photons, but can also include processes with true photons either from associated photon production or initial-state radiation. In both cases, there is no inherent \ETmiss in the event. Instead, the measured \ETmiss is actually the result of mismeasured hadronic activity. As described in Section~\ref{sec:QCD}, this background is estimated in an entirely data-driven way using a control region derived from a sideband of our photon ID. 

The second-largest background is the electroweak (EWK) background. This background is comprised of W$\gamma$ or W+jet events where W$\rightarrow e \nu$. There is inherent \ETmiss from the neutrino, and if the electron is misidentified as a photon, these events can mimic our signal topography. By measuring the misidentification rate in data, we can use an $e\gamma$ control sample to estimate the contribution from the EWK background. The background estimation method is described in detail in Section~\ref{sec:EWK}. 

Finally, there is an irreducible background from $Z\gamma\gamma\rightarrow\nu\nu\gamma\gamma$ events. This background is modeled via simulation (see Sec.~\ref{sec:Zgg}).

%%%%%%%%%%%%%%%%%%%%%%%%%%%%%%%%%%%%%%%

\section{QCD background}
\label{sec:QCD}

\subsection{Cross check on QCD background}
\label{sec:crossCheck}

In order to set a systematic uncertainty on the overall \ETmiss shape predicted using the \diempt reweighting method, we sought an alternate way to estimate the QCD background. This cross check relies on the assumption that the ratio of $\gamma\gamma$ events to $ff$ events should not depend sensitively on the \ETmiss. If this assumption is true, then we should be able to extrapolate from the low-\ETmiss control region to the high-\ETmiss signal region.

Figure~\ref{fig:crossCheck} shows the ratio of $\gamma\gamma$/$ff$ as a function of \ETmiss in the \ETmiss $<$ 100 GeV control region. The ratio has been fit to a constant function. Using this function, the expected number of $\gamma\gamma$ events in bin $i$ in the signal region is given by the following equation:

\begin{equation}
N_{\gamma\gamma}^i = f(\ETmiss) \times N_{ff}^i 
\end{equation}

XX Insert Figure XX

XX Table~\ref{tab:crossCheck} XX 



\subsection{Systematic uncertainties on the QCD background}
\label{sec:QCDSysUncert}



%%%%%%%%%%%%%%%%%%%%%%%%%%%%%%%%%%%%%%%

\section{Electroweak background}
\label{sec:EWK}

The subdominant background for this search is comprised of W$\gamma$ and W+jet events where W$\rightarrow e \nu$ and the electron is misidentified as a photon. This background is referred to as the electroweak (EWK) background. Unlike the QCD background, there is inherent \ETmiss in these events from the escaping neutrino. 

To estimate this background, we first calculate the rate at which electrons are misidentified as photons. This is done by comparing the invariant mass peak in a double electron sample with the invariant mass peak in a sample of events with one electron and one photon. The calculation of this "fake rate" is described in detail in Section~\ref{sec:fakeRate}. To get the final expected contribution from the EWK background in our signal region, the fake rate is used to calculate a transfer factor. By applying the transfer factor to an $e\gamma$ control sample, we are able to estimate how many of our candidate $\gamma\gamma$ events are actually events with one photon and one electron that has been misidentified as a photon. 

\subsection{Fake rate calculation}
\label{sec:fakeRate}

\subsection{Calculating EWK estimate}

\subsection{Composition of $e\gamma$ sample}

\subsection{EWK results}
\label{sec:EWKresults}

Table XX shows the final EWK estimate in the six signal region bins. There are two sources of uncertainty on the EWK estimate: the statistical uncertainty from the limited number of $e\gamma$ events observed in data, and the 30\% systematic uncertainty from the calculation of the fake rate. Both of these are displayed in Table XX.

%%%%%%%%%%%%%%%%%%%%%%%%%%%%%%%%%%%%%%%

\section{Irreducible background}
\label{sec:Zgg}

In addition to the QCD and EWK backgrounds, there is a small irreducible background from $Z\gamma\gamma\rightarrow\nu\nu\gamma\gamma$ events. Because there is inherent \ETmiss from this process, it is not included in the QCD background, and because it has two true photons, it is not included in the EWK background either. We model this background using MC simulation, and assign a 50\% uncertainty to the estimate to cover any potential mismodeling. 

%\subsection{Systematic uncertainties}
The systematic uncertainties fall into two main categories: 
those associated with one of the background estimates,
and other uncertainties used in the limit-setting procedure.

\subsubsection{Systematic uncertainties on QCD background}
\label:sec:QCDSys}

\subsubsection{Systematic uncertainties on EWK background}
\label{sec:EWKSys}

\subsubsection{Other sources of systematic uncertainties}
\label{sec:otherSys}
