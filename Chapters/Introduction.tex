\chapter{INTRODUCTION}

The Large Hadron Collider (LHC) at the European Organization for Nuclear Research (CERN) is the largest machine in the world and an incredible feat of engineering and science. This machine is used to study elementary particles and their interactions. At the LHC, proton-proton collisions occur at center-of-mass energies up to 13 TeV. 

Chapter~\ref{chap:theory} will describe the standard model and its supersymmetric extensions, focusing primarily on the phenomenology of the standard model and how it relates to this analysis. 
The Standard Model of particle physics describes fundamental particles and their interactions to an incredible level of precision. This is described in more detail in  Chapter XX. 

The CMS detector and various subsystems will be described in Chapter~\ref{chap:Detector}. Finally, Chapter XX will include a discussion of the results of this analysis and suggestions for future work.

Introduction

Chapter 1: Theory Motivation
	Standard Model
	Why we need BSM
	SUSY
	GMSB
Chapter 2: LHC
	About CERN
	How the accelerator works
Chapter 3: CMS Detector

Chapter 4: Object Reconstruction

Chapter 5: Trigger

Chapter 6: Event Selection
	Including object cleaning
	