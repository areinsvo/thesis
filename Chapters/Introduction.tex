\chapter{INTRODUCTION}

The standard model of particle physics is one of the biggest success stories of modern physics. It provides a framework for understanding the basic building blocks of matter: the gauge bosons, quarks, leptons, and lastly the Higgs boson, discovered in 2012. The standard model 
has incredible predictive power to describe the interactions of these elementary particles. Over the last several decades, progressively more complex accelerators and detectors have been used to probe the standard model and to search for evidence of new physics. At the same time, theorists have made progress understanding the fundamental limitations of the theory and have proposed many potential extensions.

The state-of-the-art tool to explore the standard model and test new models is the Large Hadron Collider (LHC) at the European Organization for Nuclear Research (CERN), the largest machine in the world and a remarkable feat of engineering and science. The LHC is a proton-proton ($p$-$p$) collider that can reach center-of-mass energies up to 13 TeV. The Compact Muon Solenoid (CMS) detector at the LHC is an all-purpose detector that measures the properties of particles produced in the collisions. 

One of the most active research areas of the CMS Collaboration are searches for supersymmetric extensions of the standard model. Supersymmetry (SUSY) addresses several known limitations of the standard model, 
including solving the hierarchy problem and providing potential dark matter candidates. 
In this dissertation, a search is presented for new physics 
in final states with two photons and significant missing transverse momentum. 
The results are interpreted as a search for SUSY 
appearing in the guise of gauge-mediated supersymmetry breaking (GMSB) models.

Chapter~\ref{chap:theory} will review the standard model and its supersymmetric extensions, focusing primarily on the phenomenology of the standard model and GMSB as it relates to this analysis. The LHC accelerator complex will be described in Chapter~\ref{chap:LHC}, and the 
CMS detector and subsystems will be described in Chapter~\ref{chap:Detector}. Chapters~\ref{chap:Trigger} and~\ref{sec:EventReconstruction} will outline the CMS trigger system and reconstruction algorithms, respectively. 

The primary analysis methods will be discussed in Chapters~\ref{chap:EventSelect} and \ref{chap:DataAnalysis}.
Finally, Chapter~\ref{chap:Results} will include the results and interpretations. No excess above the 
expected standard model backgrounds is observed, and limits are set on two simplified GMSB models.