\chapter{SUMMARY AND FUTURE PLANS}
\label{chap:Conclusions}

\section{Future improvements}
\label{sec:future}

There are a variety of ways in which this analysis could be improved in
future versions. Looking at Table~\ref{tab:QCDSys}, it is obvious that 
our uncertainties are dominated by the limited statistics of the $ff$ control 
sample. To acquire more statistics, it would be necessary to move to a
different trigger, since our fake definition must be at least as tight as 
the online trigger requirements. Looser ID requirements often
come at the cost of higher photon \pT thresholds. Because we 
have already pushed to high gluino and squark masses, however,
tighter \pT thresholds would probably still be feasible.

The largest potential improvement would come from using more than 
one discriminating variable. Looking at the Feynman diagrams of 
Figure~\ref{fig:gluinoSquarkDecay}, one striking feature is the 
number of expected jets in the final states of both models. We 
currently do not use any jet information in the definition of our 
signal region. By using both \ETmiss and a hadronic variable 
such as $H_T$ (the scalar sum of all visible energy in the event), 
we could achieve a better separation of signal and background.
This was studied at length using the 2016 data, but we faced issues
with correlations between variables. Our efforts are summarized
in Appendix~\ref{app:2D}.

\section{GMSB combination paper}
\label{sec:combo}
Efforts are ongoing to perform a combined analysis in the diphoton,
photon plus lepton, and inclusive single photon channels~\cite{PhotonHT, PhotonMET} to set limits on 
more realistic GMSB models. The simplified models used in this 
analysis assumed 100\% branching fraction of $\neutralino \rightarrow \gamma\gravitino$,
but in more complete GMSB models, \neutralino could also decay to a Higgs boson 
or a $Z$ boson. Additionally, in many GMSB models the lightest chargino $\widetilde{\chi}^\pm_0$
is almost as light as the lightest neutralino. We refer to this as a co-NLSP. In such cases, 
we can end up with $W^\pm$ bosons in the final state as well. 

Combining the various photonic final states is not as easy as simply adding more 
channels to the likelihood of Equation~\ref{equ:likelihood}. The first reason for this is
that the signal regions of each analysis are not exclusive. A high-\ETmiss event with two photons 
and a lepton would be counted in both the diphoton channel and the photon plus lepton channel.
The second complication is that there are correlations
in the background estimation methods of the different channels.
For instance, one of the backgrounds of the 
photon plus lepton analysis includes events where an electron is misidentified as a photon.
This is similar to our EWK background and is modeled in an equivalent way, leading to 
correlations in our uncertainties. These correlations need to be quantified and understood 
in order to set accurate limits.

Both of these difficulties 
make it impossible to perform a combination based on the information 
in the published papers alone.
Studies are currently being performed to optimize which channels should
lay claim to the overlapping regions of phase space 
and to quantify the correlations between the methods. The goal is to have
results approved in time for the 2018 summer conferences.

\section{Conclusions}
\label{sec:conclusions}
In this dissertation, a search for gauge-mediated supersymmetry breaking (GMSB) in events with two photons and missing transverse momentum was described. The analysis was performed with 35.9 \fbinv of data taken with the CMS detector in 2016. Fully data-driven background estimations were developed for the two primary backgrounds. For the QCD background, the background prediction was taken from a double fake control sample that had been reweighted by the $\gamma\gamma/ff$ \diempt distribution to correct for differences in hadronic activity. For the EWK background, we first calculated the rate at which electrons were misidentified as photons by looking at the invariant mass spectrum in an $ee$ sample and an $e\gamma$ sample. This rate was then applied to an $e\gamma$ control sample satisfying the same criteria as the candidate $\gamma\gamma$ sample. Additionally, there is a small background from $Z\gamma\gamma\rightarrow\nu\nu\gamma\gamma$ events that was modeled in simulation. 

No evidence for GMSB was observed, and limits were placed on the masses of supersymmetric particles. Results were interpreted in the context of two simplified models, one assuming squark pair production and the other assuming gluino pair production. Gluino masses below 1.90 TeV and squark masses below 1.62 TeV are excluded at a 95\% confidence level. This is an improvement of 350 GeV for gluino masses and 250 GeV for squark masses compared to the 2015 CMS results.