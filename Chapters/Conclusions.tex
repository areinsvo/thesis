\chapter{CONCLUSIONS}
\label{chap:Conclusions}

In this dissertation, a search for gauge-mediated supersymmetry breaking (GMSB) in events with two photons and missing transverse momentum was described. The analysis was performed with 35.9 \fbinv of data taken with the CMS detector in 2016. Fully data-driven background estimations were developed for the two primary backgrounds. For the QCD background, the background prediction was taken from a double fake control sample that had been reweighted by the $\gamma\gamma/ff$ \diempt distribution to correct for differences in hadronic activity. For the EWK background, we first calculated the rate at which photons were misidentified as photons by looking at the invariant mass spectrum in an $ee$ sample and an $e\gamma$ sample. This rate was then applied to an $e\gamma$ control sample satisfying the same criteria as the candidate $\gamma\gamma$ sample. Additionally, there is a small background from $Z\gamma\gamma\rightarrow\nu\nu\gamma\gamma$ events that was modeled in simulation. 

No evidence for GMSB was observed, and limits were placed on the masses of supersymmetric particles. Results were interpreted in the context of two simplified models, one assuming squark pair production and the other assuming gluino pair production. Gluino masses below 1.90 TeV and squark masses below 1.62 TeV are excluded at a 95\% confidence level. This is an improvement of 350 GeV for gluino masses and 250 GeV for squark masses compared to the 2015 CMS results.