\chapter{TRIGGER SYSTEM}
\label{chap:Trigger}

\section{Overview of CMS trigger system}
\label{sec:trigOverview}
Many more collisions occur within the LHC than can be written out to tape and stored for future analysis. At the four interaction points, including inside the CMS detector, bunches of protons collide every 25 ns. For every bunch crossing, there are on average 25 ``soft-scatter" collisions (XX Look up this number XX). This corresponds to an overall event rate of XX. Given that each event produces approximately 1 MB of data, it is impossible to store every event or even the majority of events. This has been especially true in recent years, with the LHC continually breaking new records for instantaneous luminosity. 

To solve this problem, the CMS detector includes a sophisticated trigger system to determine which events make it into the XX\% that gets written to tape and eventually analyzed. The CMS trigger system is divided into two steps: the Level 1 (L1) trigger which uses customized hardware located in the detector cavern, and the software-based High Level Trigger (HLT). The L1 trigger reduces the rate from 1 GHz to approximately 100 kHz, and the HLT makes the final decisions necessary to reduce the rate to 1 kHz, the maximum amount that can be written to tape and stored. Both steps of the CMS trigger are described in more detail below.

\subsection{Level 1 Trigger}
\label{sec:L1}
The L1 trigger reduces the rate from 1~GHz to 100 kHz. 

\subsection{High Level Trigger}
\label{HLT}


\section{Analysis triggers}
\label{sec:trigRequirements}

\section{Trigger efficiency}
\label{sec:trigEff}
An important input to the analysis is the overall trigger efficiency. Due to the similarity of the ECAL response 
to electrons and photons, the trigger efficiency can be calculated from 
$Z\rightarrow ee$ events in data using the tag-and-probe method (CITE XX). In this method, two electron candidates
are required. One serves as the ``tag" and has to pass electron identification criteria. The second electron candidate serves
as the ``probe" and has to satisfy the same selection criteria as our offline photon identification requirements (see Section~\ref{XX}). 
In order to ensure a high purity of electromagnetic objects, the invariant mass of the di-electron system must be between 75 and 105~GeV. 

Control trigger

The efficiency of the HLT path or trigger filter that is being studied is given by the following equation, where $N_{total}$ is the total number of tag-and-probe pairs 
passing all requirements, and $N_{pass}$ is the number of tag-and-probe pairs in which  the probe passes the trigger filter.
\begin{equation}
 \epsilon_{trig} = N_{pass} / N_{total}
\end{equation}

Because our analysis trigger is seeded by the OR of multiple SingleEG and DoubleEG L1 seeds, the names of which changed over the course of the 2016 data-taking period, calculating the L1 efficiency on its own proved tricky. Instead, it was simpler to calculate the total efficiency with which photon candidates pass both the L1 seed and the leading leg of the HLT path. This efficiency as a function of photon \pt is shown in Figure~\ref{XX}. The efficiency was fit to an error function to calculate the overall efficiency at the plateau. For photon \pt $>~40$ GeV, the leading filter is XX\% efficient.

XX Insert figure XX

Tag and probe objects for the trailing leg efficiency must pass the same set of requirements as those used in the leading leg efficiency calculation, with the additional requirement that the tag must pass the leading filter. This requirement comes from how HLT paths are run. Filters are processed sequentially, and if an event fails one filter, the subsequent filters are skipped. Figure~\ref{XX} shows the efficiency of the trailing filter as a function of photon \pt. 

XX Insert figure XX

Finally, we calculated the efficiency of the trigger with respect to the diphoton invariant mass. For this calculation, we required two photons passing our analysis selection criteria and the trailing leg of the trigger and one photon passing the leading leg of the trigger. The efficiency was given by the number of diphoton events passing the full HLT path over the total number of diphoton events passing our requirements. The efficiency of the trigger as a function of invariant mass is shown in Figure~\ref{XX}.

XX Insert figure XX

\subsection{Efficiency of di-electron control sample}
\label{sec:eeEff}
One of the important control samples for our analysis is made up of $Z\rightarrow ee$ events. Due to the invariant mass requirement on the primary analysis, however, a second trigger was used to collect events for the di-electron control sample. This trigger shares all of the same requirements as the primary trigger, with two exceptions: the invariant mass of the two electromagnetic objects is required to be greater than 55 GeV rather than 95 GeV, and both electromagnetic objects are required to be matched to a pixel seed. The pixel seed match requirement affects the efficiency of the trailing leg of the control trigger. This efficiency is shown in Figure~\ref{XX}.

XX Insert figure XX


