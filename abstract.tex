This dissertation describes a search for gauge-mediated supersymmetry breaking (GMSB) 
in final states with two photons and missing transverse momentum, $E_{\mathrm{T}}^{\text{miss}}$. 
Supersymmetry (SUSY) is an attractive extension to the standard model of particle physics. 
It addresses several known limitations of the standard model, 
including solving the hierarchy problem and providing potential dark matter candidates. 
In GMSB models, the lightest supersymmetric particle is the gravitino, and the next-to-lightest supersymmetric particle is often taken to be the neutralino. The neutralino decays predominantly to a photon and a gravitino, leading to final state signatures with one or more photons and significant $E_{\mathrm{T}}^{\text{miss}}$. 
The search described in this dissertation was performed using proton-proton collisions at a center-of-mass energy $\sqrt{s} = 13$ TeV. The data were collected with the Compact Muon Solenoid (CMS) detector at the CERN LHC in 2016. The total integrated luminosity of the data set is 35.9 fb$^{-1}$. 
After providing an overview of the standard model and the theory behind GMSB models, I will introduce the LHC and the CMS detector, describing in detail how the triggering and event reconstruction are performed.
More importantly, the standard model backgrounds and corresponding data-driven estimation methods will be outlined.
No excess above the expected standard model backgrounds is observed, and 
limits were placed on the masses of SUSY particles in two simplified GMSB models. 
Gluino masses below 1.90 TeV and squark masses below 1.62 TeV are excluded at a 95\% confidence level.
